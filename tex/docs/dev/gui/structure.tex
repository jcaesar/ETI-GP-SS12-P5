The GUI is implemented solely in Java, utilizing the Swing Architecture. The project is structured according to the Model–View–Controller pattern. It consists of three packages (\texttt{data}, \texttt{view} and \texttt{controller}), separating the data from the interface, connected through the \texttt{controller}. The package \texttt{data} is responsible for reading the data files and providing a comprehensive interface to retrieve the data. The package \texttt{view} is responsible for displaying the data to the user. It communicates with the data only via the \texttt{controller} (and vice versa). Therefore the \texttt{controller}'s purpose is to manage the entire data flow of the software. It sends requested data to the user interface or requests new data to be loaded from a file. 

\subsection{Class Responsibilities - Package \texttt{view}}
\includegraphics[width=350px]{gui/classresp2.png}
\newpage 
\includegraphics[angle=90,height=700px]{gui/classresp1.png}
\subsection{Informal Class Diagramm}
\includegraphics[angle=90,height=560px]{gui/comm.png} 