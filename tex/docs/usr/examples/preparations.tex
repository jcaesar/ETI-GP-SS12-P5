Suppose we have a square matrix \texttt{a} of size \texttt{M} $\times$ \texttt{N} which we want to trace, then the following code snippet will tell the tool to record and analyze all accesses to the memory area where \texttt{a} is stored:
\begin{lstlisting}
#include <valgrind/mctracer.h>
// ...
   SSIM_MATRIX_TRACING_START(a, M, N, sizeof(double), "a");
   // ... (crucial matrix accesses that are to trace)
   SSIM_MATRIX_TRACING_STOP(a);
\end{lstlisting}
The commands in lines 3 and 5 are Valgrind client requests by means of which the debuggee can communicate with the tool (Section \ref{vgclientcalls}).
%The syntax of these requests is as follows:
%
%\begin{lstlisting} [numbers=none]
%SSIM_MATRIX_TRACING_START(void* addr, int m, int n, 
%                          size_t ele_size, char* name);
%\end{lstlisting}
%signals the beginning of a trace of all accesses on the memory area starting at address \texttt{addr} representing a \texttt{m}-by-\texttt{n} matrix with entries of size \texttt{ele\_size}. When tracing multiple matrices, the argument \texttt{name} should be a unique identifier for the traced matrix, as the user interface will display the specified name for each traced matrix.
%With the request
%\begin{lstlisting} [numbers=none]
%SSIM_MATRIX_TRACING_STOP(void* addr);
%\end{lstlisting}
%you can stop the recording of accesses on the referenced matrix.

Valgrind client requests for starting and stopping the tracing must be called for each matrix.

\subsubsection*{Comparing multiple implementations}
For the problems in the previous examples, there were different solution approaches. Each of these can be encapsuled in a method of its own. Suppose we have the two methods
\begin{lstlisting} [numbers=none]
void method1(double* a) { /* ... */ }
void method2(double* a) { /* ... */ }
\end{lstlisting}
which solve a certain problem on a \texttt{N}-by-\texttt{N} matrix \texttt{a}; then we have to use the requests once for each method:
\begin{lstlisting}
init(a); // initializes the entries of a
SSIM_MATRIX_TRACING_START(a, N, N, sizeof(double), 
                          "method1 - a");
method1(a);
SSIM_MATRIX_TRACING_STOP(a);

SSIM_FLUSH_CACHE();
init(a);
SSIM_MATRIX_TRACING_START(a, N, N, sizeof(double), 
                          "method2 - a");
method2(a);
SSIM_MATRIX_TRACING_STOP(a);
\end{lstlisting}
As both methods operate on the same matrix, calling \texttt{method2} directly after \texttt{method1} (and after the repeated initialization) could distort the memory access statistics if the cache still holds some matrix entries that are later used again in \texttt{method2}. (We assume that the \texttt{init} procedure does not flush the cache or resets all matrix entries.) Therefore we need to flush the entire cache between the two method calls in line 7. %, which can be done via another Valgrind client request:
%\begin{lstlisting} [numbers=none]
%SSIM_FLUSH_CACHE();
%\end{lstlisting}

\subsubsection*{Further implementation notes}
If you have many alternative implementations which have the same method signature, it may prove useful to define a function pointer to such an implemented method and use that in generic function calls. This makes the different methods exchangable and comparable.

