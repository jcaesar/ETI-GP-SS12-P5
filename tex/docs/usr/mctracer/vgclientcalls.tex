\lstset{language=C}

The following valgrind client calls are provided by mctracer, to use them, you have to
\begin{lstlisting} [numbers=none]
#include <valgrind/mctracer.h>
\end{lstlisting}
in your debuggee source.

\subsubsection{Start Tracing a Matrix}
When you want to start simulating cache actions on a matrix, you have to use this call.
\begin{lstlisting} [numbers=none]
SSIM_MATRIX_TRACING_START(addr, m, n, element_size, name)
\end{lstlisting}
\begin{description}
\item[\texttt{addr}] The address of the first element of your array.
\item[\texttt{m},\texttt{n}] The dimensions of the matrix.
\item[\texttt{element\_size}] The size, in bytes, of one element.
\item[\texttt{name}] A name you are free to specify to identify your matrix later.
\end{description}


\subsubsection{Stop Tracing a Matrix}
When you want to stop simulating cache actions on a matrix (because you free it or it's not going to be on stack anymore), you have to use this call.
\begin{lstlisting} [numbers=none]
SSIM_MATRIX_TRACING_STOP(addr)
\end{lstlisting}
\texttt{addr} can be any address inside the matrix' memory area.

\subsubsection{Flush the Cache}
Sometimes, you want to run multiple things in one program and trace them separately. To avoid interference you can use this call to reset the cache simulator.
\begin{lstlisting} [numbers=none]
SSIM_FLUSH_CACHE
\end{lstlisting}

