\begin{frame}{Valgrind}
	\begin{itemize}
		\item Valgrind ist ein Framework um dynamische Analyseprogramme (Tools) zu entwickeln.
		\pause \item Valgrind ist eine virtuelle Maschine und JIT Compiler.
		\pause \item Valgrind verwendet Dynamic Binary Translation (DBT) 
		\pause \item Tools führen Analyse und/oder Instrumentierung durch
		\pause \item Einfachste Tools koennen mit nur vier Funktionen implementiert werden.
		\pause \item{\bf Tools:} Memcheck, Cachegrind, Callgrind, {\bf McTracer} 
	\end{itemize}
\end{frame}

\subsection{Valgrind Workflow}

\begin{frame}{Valgrind Workflow}
	\begin{enumerate}
		\item Das Programm wird eingelesen und in die Vex Intermediate Representation (UCode) ueberfuehrt.
		\pause \item IRSB's (IR Super Block) werden an das Tool uebergeben. 
		\pause \item Das Skin instrumentiert/analysiert die IRSB's.
		\pause \item Die IRSB's werden zu Maschinencode kompiliert und nativ ausgefuehrt. 
	\end{enumerate}
\end{frame}

\begin{frame}{Darstellung des Maschinencodes in Vex IR}
	\lstset{frame=single}
	\lstinputlisting[language=C, basicstyle=\tiny]{mctracer/irsb.c}
\end{frame}

\begin{frame}{Instrumentierung}
	\lstset{frame=single}
	\lstinputlisting[language=C, basicstyle=\tiny]{mctracer/instrument.c}	
\end{frame}

\subsection{Valgrind Client Requests}

\begin{frame}{Valgrind Client Requests}
	\begin{itemize}
		\item Client Requests erlauben dem analysierten Programm mit Valgrind bzw. dem Tool zu kommunizieren.
		\pause \item Client Requests sind ein Trapdoor Mechanismus.
		\pause \item Werden bei McTracer verwendet um die zu analysierenden Datenstrukturen festzulegen.
	\end{itemize}

	\pause

	\lstset{frame=single}
	\lstinputlisting[language=C, basicstyle=\tiny]{mctracer/client_example.c}
\end{frame}

\begin{frame}{Valgrind Client Requests}
	\lstset{frame=single}
	\lstinputlisting[language=C, basicstyle=\tiny]{mctracer/client_impl.c}	
\end{frame}

\begin{frame}{Valgrind Client Request Internals}
	\lstset{frame=single}
	\lstinputlisting[language=C, basicstyle=\tiny]{mctracer/client_request_valgrind.c}	
\end{frame}
