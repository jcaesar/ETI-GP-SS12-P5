\begin{frame}{Valgrind}
	\begin{itemize}
		\item Valgrind ist ein Framework um dynamische Analyseprogramme (Tools) zu entwickeln.
		\pause \item Valgrind ist eine virtuelle Maschine und JIT Compiler.
		\pause \item Valgrind verwendet Dynamic Binary Translation (DBT) 
		\pause \item Tools führen Analyse und/oder Instrumentierung durch.
		\pause \item{\bf Tools:} Memcheck, Cachegrind, Callgrind, {\bf McTracer} 
	\end{itemize}
\end{frame}

\subsection{Valgrind Workflow}

\begin{frame}{Valgrind Workflow}
	\begin{enumerate}
		\item Das Programm wird eingelesen und in die Vex Intermediate Representation (Vex IR) ueberfuehrt.
		\pause \item IRSB's (IR Super Block) werden an das Skin uebergeben. 
		\pause \item Das Skin instrumentiert/analysiert die IRSB's.
		\pause \item Die IRSB's werden zu Maschinencode kompiliert und nativ ausgefuehrt. 
	\end{enumerate}
\end{frame}

\begin{frame}{Darstellung des Maschinencodes in Vex IR}
	\lstset{frame=single}
	\lstinputlisting[language=C, basicstyle=\tiny]{mctracer/irsb.c}
\end{frame}

\begin{frame}{Instrumentierung}
	\lstset{frame=single}
	\lstinputlisting[language=C, basicstyle=\tiny]{mctracer/instrument.c}	
\end{frame}

\begin{frame}{Valgrind Client Request}
	\begin{itemize}
		\item Client Requests erlauben dem analysierten Programm mit Valgrind bzw. dem Tool zu kommunizieren.
		\pause \item Client Requests sind ein Trapdoor Mechanismus.
		\pause \item Werden bei McTracer verwendet um die zu analysierenden Datenstrukturen festzulegen.
	\end{itemize}

	\pause

	\lstset{frame=single}
	\lstinputlisting[language=C, basicstyle=\tiny]{mctracer/client_example.c}	
\end{frame}
