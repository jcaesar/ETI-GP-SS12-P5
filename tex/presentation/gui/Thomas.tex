\subsection{Absolute Matrixdarstellung}
\begin{frame}{Absolute Matrixdarstellung}
	\begin{block}{Anforderungen}
		\pause
		\begin{itemize}[<+->]
			\item Repräsentation aller Matrixelemente, farbliche Codierung als Indikator für Hits/Misses-Verhältnis\\\pause
				$\Rightarrow$ Graphics.setColor(Color c)\\
				$\Rightarrow$ Graphics.drawRect(int x, int y, int width, int height)\pause
			\item Zoomen und Verschieben\\\pause
			$\Rightarrow$ JScrollPane\\
			$\Rightarrow$ Affine Abbildung
		\end{itemize}
	\end{block}
\pause
\end{frame}

\subsection{Affine Abbildung}
\begin{frame}{Affine Abbildung}
	\begin{block}{Definition}
		Abbildung zwischen 2 (affinen) Räumen, wobei
		\pause
		\begin{itemize}[<+->]
			\item Kollinearität
			\item Parallelität
			\item Teilverhältnisse
		\end{itemize}
		bewahrt bleiben oder gegestandslos werden.
	\end{block}
\pause
$\Rightarrow$ Affine Abbildung von Raum mit ursprünglich gezeichneter Matrix in gezoomte/verschobene Darstellung
\end{frame}

\begin{frame}
\frametitle{Affine Abbildung - Implementierung}
\pause
\scriptsize$\underbrace {\left(
   \begin{array}{ccc}
     1 & 0 & scale*d_{x} \\
     0 & 1 & scale*d_{y} \\
     0 & 0 & 1
   \end{array}
\right)}_{Verschiebung}\pause*\underbrace {\left(
   \begin{array}{ccc}
     scale & 0 & 0 \\
     0 & scale & 0 \\
     0 & 0 & 1
   \end{array}
\right)}_{Zoom}=\pause\left(
   \begin{array}{ccc}
     scale & 0 & scale*d_{x} \\
     0 & scale & scale*d_{y} \\
     0 & 0 & 1
   \end{array}
\right)$
\vspace{\baselineskip}
\vspace{\baselineskip}
\pause
\lstset{frame=single}
\lstinputlisting[language=JAVA]{gui/affine.java}
\end{frame}