\subsection{Für was soll ETIS gut sein?}
\begin{frame}
\begin{block}{Für was soll ETIS gut sein?}
\begin{itemize}[<+->]
\pause\item Grundidee ist die verbesserte Ausnutzung des Caches
\item Diese Ausnutzung ist wichtig aufgrund des "Von-Neumann-Flaschenhals"
\item Cache ist ein schneller Zwischenspeicher, der vorausschauend Daten aus dem Hauptspeicher läd
\item ETIS soll dem Programmierer helfen den Cache optimal auszunutzen
\end{itemize}
\end{block}
\end{frame}

\subsection{Wie hilft ETIS dem Programmierer?}
\begin{frame}
\begin{block}{Wie hilft ETIS dem Programmierer?}
\begin{itemize}[<+->]
\pause\item ETIS beinhaltet eine graphische Oberfläche, die dem Benutzer Informationen zum Cache Verhalten liefert
\item Darstellung von Hits und Misses in der grafischen Oberfläche als relative Zugriffe, als auch in der absoluten Matrix
\item Finden von Mustern, z.B. durchlaufen in einem Kreis
\item ETIS ist konzipiert für die Simulation von Matritzenoperationen, wie z.B. Matrixmultiplikation
\end{itemize}
\end{block}
\end{frame}

\subsection{Technischer Funktionsablauf}
\begin{frame}
\begin{block}{Technischer Funktionsablauf}
\begin{itemize}[<+->]
\pause\item ETIS führt die Programme auf einem Cache-Simulator durch
\item Gewinnt Informationen über das Verhalten der Programme auf den Cache und kann die Hits und Misses den Operationen zuordnen
\item Suchen von Mustern in den gewonnenn Informationen und schreiben eines Files für die Oberfläche
\item Graphische Oberfläche entkoppelt, als Eigenständiges Java Programm, welches die Informationen als Datei erhält (*.etis)
\end{itemize}
\end{block}
\end{frame}

