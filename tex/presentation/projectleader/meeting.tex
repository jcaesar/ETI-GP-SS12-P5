\begin{frame}
\tiny{\tableofcontents[currentsection,hideallsubsections]}
\end{frame}

\subsection{Teamaufteilung}
\begin{frame}{Teamaufteilung}
\begin{block}{Teamaufteilung}
\begin{itemize}[<+->]
\pause\item 3 Personen beschäftigen sich mit dem MCTracer und SimpleSim
\item 2 Personen sind für die graphische Oberfläche zuständig
\item 1 Person kümmert sich um die Beispielprogramme
\item 1 Person fungiert als Springer, welcher anfangs bei der Projektleitung und danach bei der GUI mithilft
\item 1 Person ist für die Projektleitung verantwortlich
\end{itemize}
\end{block}
\end{frame}

\begin{frame}{Zuteilung der Personen}
\begin{block}{Zuteilung der Personen}
\begin{itemize}[<+->]
\pause\item Teilprojekt MCTracer und SimpleSim: Nils Kunze, Jakob Buchgraber, Julius Michaelis
\item Teilprojekt GUI: Philip Becker-Ehmck, Thomas Breier, Simon Wimmer (Springer)
\item Teilprojekt Beispielprogramme: Matthias Brugger
\item Teilprojekt Projektleitung: Dominik Durner, Simon Wimmer (Springer)
\end{itemize}
\end{block}
\end{frame}

\subsection{Aufgaben der Teilprojekte}

\begin{frame}{Aufgaben der Teilprojekte}
\begin{block}{Teilprojekt MCTracer und SimpleSim}
\begin{itemize}[<+->]
\pause\item Valgrind, SimpleSim und MCTracer verstehen und für unsere Anwendung bearbeiten
\item Zuordnung der Speicherzugriffe für die detaillierte Auswertung
\item Analyse der Zugriffe und Finden von Mustern, nach denen die Matrix durchlaufen wird
\item Bereitstellung der Daten für die GUI mithilfe des \glqq etis\grqq-Dateiformats
\end{itemize}
\end{block}
\end{frame}

\begin{frame}{Aufgaben der Teilprojekte}
\begin{block}{Teilprojekt GUI}
\begin{itemize}[<+->]
\pause\item Grafische Darstellung der Zellen der Matrix
\item Darstellung der relativen Zugriffe
\item Ausgabe der Statistiken über Hits und Misses
\item Darstellung der Sequenzen und durchlaufenen Muster
\end{itemize}
\end{block}
\end{frame}

\begin{frame}{Aufgaben der Teilprojekte}
\begin{block}{Teilprojekt Beispielprogramme}
\begin{itemize}[<+->]
\pause\item Analyse der Beispielprogramme
\item Bearbeiten der Beispielprogramme, um genaue Informationen an den MCTracer liefern zu können
\end{itemize}
\end{block}
\end{frame}

\begin{frame}{Aufgaben der Teilprojekte}
\begin{block}{Teilprojekt Projektleitung}
\begin{itemize}[<+->]
\pause\item Erstellen der Aufgabenstellung
\item Koordination des gesamten Teams und Hilfe bei Schnittstellenfindung
\item Einberufung der Meetings
\item Vorbereitung der Abschlusspräsentation
\item Ansprechpartner für die einzelnen Teilprojekte
\end{itemize}
\end{block}
\end{frame}

\subsection{Meetings}

\begin{frame}{Meetings}
\begin{block}{Meetingübersicht}
\begin{itemize}[<+->]
\pause\item Erste Besprechung am 04.05.2012
\item Meeting am 18.05.2012
\item Meeting am 01.06.2012
\item Vorbesprechung am 13.06.2012
\item Meeting am 01.08.2012
\item Abschlusspräsentation am 08.08.2012
\end{itemize}
\end{block}
\end{frame}

\begin{frame}{Meetings}
\begin{block}{Erste Besprechung}
\begin{itemize}[<+->]
\pause\item Herr Weidendorfer stellt uns vor, was genau zu tun ist :)
\item Überlegung, ob relative Zugriffe oder absolute Zugriffe anzuzeigen sind
\item Umfang der Aufgabenstellung
\item Termin zur Zwischenpräsentation (13.06.2012) wird vereinbart
\item Benötigt wird eine Benutzer- und eine technische Dokumentation
\item Zum Ende des Projektes soll eine Abschlusspräsentation gehalten werden
\end{itemize}
\end{block}
\end{frame}

\begin{frame}{Meetings}
\begin{block}{Meeting am 18.05.2012}
\begin{itemize}[<+->]
\pause\item Ziel: Fertigstellung des Projektes noch in diesem Semester
\item Zuteilung der Teams mit Wunschberücksichtigung
\item Grobe Anforderungen werden formuliert, die die Teams bearbeiten müssen
\item Teaminterne Deadlines zur besseren Übersicht werden ausgemacht
\end{itemize}
\end{block}
\end{frame}

\begin{frame}{Meetings}
\begin{block}{Meeting am 01.06.2012}
\begin{itemize}[<+->]
\pause\item Teaminterne Vorstellung der Zwischenergebnisse
\item Verbesserungsvorschläge und Erarbeiten weiterer Anforderungen
\end{itemize}
\end{block}
\end{frame}

\begin{frame}{Meetings}
\begin{block}{Vorbesprechung am 13.06.2012}
\begin{itemize}[<+->]
\pause\item Erste offizielle Vorführung des Programmes
\item Erarbeiten von Verbesserungen mithilfe von Herrn Weidendorfer
\item Herr Weidendorfer gibt Vorschläge für zusätzliche Erweiterungen, falls genügend Zeit besteht (z.B. Mustererkennung)
\item Verschieben der Abschlusspräsentation ans Ende des Semesters (nach der Prüfungszeit)
\end{itemize}
\end{block}
\end{frame}

\begin{frame}{Meetings}
\begin{block}{Meeting am 01.08.2012}
\begin{itemize}[<+->]
\pause\item Kurze Vorstellung der aktuellen Ergebnisse
\item Erarbeiten einer Timetable für die Abschlusspräsentation
\item Letzte Anforderungen werden umgesetzt
\item Für die Präsentation und spätere Dokumentation werden Vorbereitungen getroffen
\end{itemize}
\end{block}
\end{frame}